\documentclass[11pt,twoside]{../Filofax-Template/filofax2}
%\usepackage{fbox}
%\usepackage{array}

\begin{document}

%\setlength\unitlength{1cm}
\begin{ffpage}{\Large \bf Page specification}
%\hrule\vspace{3pt}
%\reversemarginpar
%\marginpar{\framebox(\marginparwidth,\textheight){a}}

textwidth \the\textwidth\\
textheight \the\textheight\\
paperwidth \the\paperwidth\\
paperheight \the\paperheight\\

voffset \the\voffset \\ 
hoffset \the\hoffset \\

marginparwidth \the \marginparwidth \\
marginparsep \the \marginparsep \\

oddsidemargin \the\oddsidemargin  \\
evensidemargin \the\evensidemargin \\


topmargin \the\topmargin \\
leftmargin \the\leftmargin \\
topskip \the\topskip \\
headheight \the\headheight \\
headsep \the\headsep \\
marginparwidth \the\marginparwidth \\
%\the\marginsep
%\the\twosides

voffset \the\voffset \\
hoffset \the\hoffset \\


%\reversemarginpar
%\marginpar{\framebox(\marginparwidth,\textheight){}}


%\rule{\textwidth}{3pt}
%\vspace{9pt}
\end{ffpage}


\end{document}